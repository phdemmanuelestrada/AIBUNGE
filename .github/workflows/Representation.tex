\title{Capítulo — ''Representation'' (traducción y codificación)}
\author{Equipo}
\date{\today}
\maketitle

\noindent DEFINITION 3.2 Let $T$ be a theory about entities of kind $K$ and call $S(x)$
the collection of possible states (in example the state space) of thing $x \in K$. Further,
call 
\[
S=\bigcup\limits_{x\in K}S(x)
\]
the union of the state spaces of all the members of $K$
(so that an arbitrary state of an arbitrary element of $K$ is in $S$). Then $T$
is said to be a representation of K's iff there exists a function ,;, : $S ~ T$
assigning to every state s E S a statement t E T. In this case $'t = ,;$, (s), is
read 't represents s'.
